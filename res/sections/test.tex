\subsection{Tests}
\label{sec:tests}

We executed some tests to study the maximal throughput and the size of the
blockchain with different configurations. To analyze the scalability of a
permission-less blockchain system such as Ethereum, one should either rely on
simulation~\cite{bib:securityAndScalabilityPoW} or run tests using thousand of
nodes~\cite{bib:securityAndScalabilityPoW, bib:algorand}. We have neither a
simulation of an Ethereum system nor so many resources. Therefore, we took
inspiration from Blockbench~\cite{blockbench}, which compares the performance
and scalability of Hyperledger\footnote{\url{https://www.hyperledger.org/}} and
Ethereum in a \emph{private} (\emph{permissioned}) scenario, that is,
considering a limited number of authenticated nodes.

We tried to use the publicly available Blockbench
repository\footnote{\url{https://github.com/ooibc88/blockbench}} but we did not
manage to configure it due to a lot of hard-coded configuration variables and
the lack of a well-written documentation. Thus, we wrote our own test and
benchmark
system\footnote{\url{https://github.com/gfornari/ethereum-test/tree/benchmark}}.


\subsubsection{Test Configuration}

To keep the configuration easy, we opted for a classic master-slave logic. The
master, i.e. the initiator and coordinator of the tests, uses the ssh protocol
to run commands on the remote machines. Similarly to~\cite{blockbench}, we
distinguish between the \emph{miner} and \emph{client} roles. The nodes of the
former type are accountable to generate new blocks while the nodes of the latter
type create and propagate transactions, and both verify the blocks\footnote{To
reduce the number of test variables we consider only full nodes. For a list of
node types we refer to~\autoref{sec:node-types}.}. We can assign multiple roles
to a single machine. In this case we run \emph{one distinct \texttt{geth}
instance} for each different role. The coordinator copies the right genesis file
in the test machines.

In each run of test, we distinguish two main phases:
\begin{enumerate*}
  \item the setup and
  \item the test itself.
\end{enumerate*}

\paragraph{Setup phase}
The setup consists in iterating on the test machines twice:
\begin{enumerate}
  \item the required ethash data structures are generated
  \item the genesis file is used to create the genesis block and initialize the
  ethereum World State.
\end{enumerate}


\subparagraph{Ethash data structures}
During the setup phase, the miners generate the DAG and the cache for the first
two epochs, while the clients generate only the caches because they are required
only to validate blocks. We generate the DAGs for the first two epochs, because
ethash uses double buffer of DAGs to grant a smooth switch between
epochs~\cite{bib:dagger-hashimoto}.

\subparagraph{Genesis file}
The genesis file contains useful information to create (deterministically) the
genesis block and the initial state. We report an extract of the genesis file we
used in~\autoref{listing:genesis}. It is simply a JSON file, which specifies
several parameters. Most of them directly specify the attributes of the block
number $0$, which are shown in \autoref{fig:world-state}. Here, we describe only
the fields that are not directly a parameter specification:
\begin{itemize}
  \item \texttt{config}: it describes the network id, and the number and hash of
  blocks that marks the entry into force of the Ethereum Improvement Proposals
  (EIP), which indicates incompatible changes in the protocol or simply a new
  version of Ethereum
  \item \texttt{alloc}: it specifies an initial allocation of Ether for the
  accounts\footnote{This possibility has been exploited for the so-called
  Initial Coin Offering (ICO) used by the Ethereum Foundation to obtain fiat
  currency to finance the project.} we use in the tests.
\end{itemize}

\begin{figure}[h!]
  \lstinputlisting[caption={An extract of the genesis file used in the tests.}, label=listing:genesis]{./res/code/genesis-file.json}
\end{figure}

Obviously, each node of the network should be initialized with the same genesis
file, otherwise the hash of the genesis block differs and the peers cannot
establish a connection with the Ethereum Wire Protocol as described
in~\autoref{sec:ethereum-wire-protocol}. Thus, the genesis file for the main
network and the official Ethereum test networks are hard-coded. To create a new
private Ethereum network it is sufficient to use a new genesis file in which
some parameters are changed. Therefore, in our genesis file we use an arbitrary
network id and nonce, so that packets of different networks are simply dropped.

The only parameters which affect the transaction throughput are the timestamp,
the difficulty and the gas limit. In the following paragraphs, we provide an in
depth description of these parameters and a justification for the values we used
in the tests.

%The values of some parameters require some
%justification.

The \textbf{timestamp} of the genesis file, which corresponds to the one of the
genesis block, affects the difficulty of the first blocks, and transitively
of all the blocks. Since, for time constraints, we want to run each test for few
minutes, we want to avoid sudden decreases in the difficulty due to a too old
timestamp. Therefore, to prevent these unwanted changes in the difficulty value,
during the second loop of the setup phase, the initiator reads its timestamp and
gives it as parameter to the peers\footnote{Before starting the test all the
peers should be configured, therefore using the timestamp of the coordinator
does not assume fine-grained coordinated clocks.}.

As already described in \autoref{sec:consensus:algorithm}, the
\textbf{difficulty} is an adaptive parameter that determines how much effort
should be invested in the creation of a new block. In our tests we used the same
hardware and same operating system in all nodes and for all miners (we
deliberately used only one thread for mining). Therefore, to find a suitable
starting value for the difficulty for the different configurations, we ran a
simulation with one, two, four and eight miners for 24 hours.
\autoref{fig:start_difficulty_raw} shows how the difficulty changed with the
different number of miners. We can notice that in our homogeneous system, the
final values of difficulties have a quasi linear dependency with the number of
miners. For the test we took as initial value the median of the last $100$
blocks. These values are represented in~\autoref{table:start-difficulty}. We
reported also the coefficient of variation to show that the values of the last
$100$ blocks are pretty stable.

\begin{figure}[h!]
  \begin{center}
    \includegraphics[width=0.8\textwidth]{./res/img/start_difficulty_all.png}
    \caption{The growth of the difficulty in the 24 hour run.}
    \label{fig:start_difficulty_raw}
  \end{center}
\end{figure}

\begin{table}[h!]
  \begin{center}
    \begin{tabular}{c | c | c}
      Number of Miners & Difficulty & Coefficient of variation \\
      \hline
      1 &  404559 & 0.0073\% \\
      2 &  821994 & 0.2330\%\\
      4 & 1711150 & 0.1458\%\\
      8 & 3409299 & 0.1742\%\\
    \end{tabular}
    \caption{The median of the last $100$ blocks, value used in the tests as
    start difficulty, and the coefficient of variations.}
    \label{table:start-difficulty}
  \end{center}
\end{table}

The \textbf{gas limit} in the genesis block determines the gas limit of the
first block. The gas limit of the subsequent blocks can be determined freely by
the miners but must be contained in a range obtained by summing and subtracting
to the gas limit a portion of itself~\cite{wood2018ethereum}. Thus, in a
relatively brief simulation the initial value is fundamental.


\paragraph{Test Phase}
After the setup phase, the real test begins by starting the execution of the
\texttt{geth} instances on the different machines. The clients emit transactions
that cause simple transfers of values that do not require the execution of
the EVM, with a predefined release time ($50$ ms), while the miners starts
collecting transactions and creating blocks. The tests are stopped after a
configurable time ($10$ minutes). For each configuration the tests are repeated
a configurable number of times ($5$) because the probabilistic nature of the
Proof-of-Work algorithm and the interaction of many systems do not guarantee
deterministic results.


\paragraph{Concrete Parameters}
We measure the throughput with different numbers of miners ($1$, $2$, $4$ and
$8$), fixing the number of clients at $16$ with two different gas limits. The
miner machine executes only one \texttt{geth} instance in miner mode, while the
client machines execute two instances in client mode. For our tests we used the
Scaleway platform\footnote{\url{https://www.scaleway.com/}}. We rent $17$ (one
master and $16$ slaves) START1-S servers, which are provided with $2$ X86 64 bit
Cores and $2$ GB of RAM, a 50 GB SSD and an internal bandwidth of $1$ Gbit/s, so
that the network would not be a bottleneck. Each server runs Ubuntu 16.04 LTS
with \texttt{geth} version 1.8.11-stable. During the tests, the master acts also
as bootstrap node. We described the role of bootstrap nodes
in~\autoref{sec:network-layer}.

The two tests we conducted differs only for the gas limit in the genesis block,
we wanted to confirm that this value influence the number of transaction
processed as described in \autoref{sec:background}.


\subsubsection{Results}

In this section we report interesting results obtained from the two tests we
conducted.

\paragraph{Maximal throughput - Low Gas Limit}
\label{sec:max-troughput}
For this test we used a gas limit of $3141592$, which corresponds to
approximately $150 \cdot 21000$ gas units\footnote{$21000$ gas units is the cost
for a transaction without EVM computations.}. The results are reported
in~\autoref{tab:max-troughput}. We notice that in this case we could not exceed
$10$ transactions per second and that the number of transactions in each
block do not exceed $135$, although there would be enough space to add other
transactions.

\begin{table}[h!]
  \centering
  \begin{tabular}{l | cccc}
    & 1 miner & 2 miners & 4 miners & 8 miners \\ \hline
    Avg throughput & 5.37 & 7.17 & 6.73 & 6.65 \\
    Avg blocks & 24.00 & 37.80 & 30.02 & 30.00 \\
    Avg txs per block & 133.77 & 114.69 & 133.61 & 133.15 \\
    Max throughput & 7.18 & 7.76 & 8.17 & 8.67 \\
    Min throughput & 3.68 & 6.83 & 5.91 & 4.91 \\
  \end{tabular}
  \caption{Average throughput and number of mined blocks on 5 runs.}
  \label{tab:max-troughput}
\end{table}


\paragraph{Maximal throughput - High gas limit}
\label{sec:max-throughput-high-gaslimit}
For this test we used a very large gas limit $10500000$ which corresponds to $500
\cdot 21000$ gas units. For reference, currently
(August 2018) the gas limit for the main Ethereum network is approximately $8$
million, which corresponds roughly to $381 \cdot 21000$ gas units. The results are
summarized in~\autoref{tab:max-troughput-high-gaslimit}. In this case with two
configurations, we could exceed $15$ transactions per second.

\begin{table}[h!]
  \centering
  \begin{tabular}{l | cccc}
    & 1 miner & 2 miners & 4 miners & 8 miners \\ \hline
    Avg throughput & 8.17 & 11.55 & 9.02 & 11.07 \\
    Avg blocks & 33.60 & 27.20 & 37.20 & 29.80 \\
    Avg txs per block & 145.25 & 262.98 & 157.15 & 226.32  \\
    Max throughput & 12.55 & 17.50 & 17.77 & 12.525 \\
    Min throughput & 6.82 & 6.83 & 6.83 & 7.14 \\
  \end{tabular}
  \caption{Average throughput and number of mined blocks on 5 runs with high gas limit.}
  \label{tab:max-troughput-high-gaslimit}
\end{table}
