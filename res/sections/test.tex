\subsection{Tests}
\label{sec:tests}

To measure the scalability of Ethereum we prepared some tests to study the
maximal throughput and the size of the blockchain with different configurations.
To study the scalability of one permission-less blockchain system such as
Ethereum one should prepare some tests using thousand of
nodes~\cite{bib:securityAndScalabilityPoW, bib:algorand}. Since we do not
dispose of so many resources we take inspiration from
Blockbench~\cite{blockbench}, which compares the performance and scalability of
Hyperledger\footnote{\url{https://www.hyperledger.org/}} and Ethereum in a
\emph{private} scenario, that is when we take into consideration a limited
number of authenticated nodes.

We tried to use the public available Blockbench
repository\footnote{\url{https://github.com/ooibc88/blockbench}} but we did not
manage to configure it, because of a lot of hard-coded configuration variables
and the lack of a well-written documentation. Therefore we desisted and wrote
our own system.


\subsubsection{Test Configuration}

To keep the configuration easy we opted for a classic master-slave logic. The
master, i.e. the initiator and coordinator of the tests, uses the ssh protocol
to run commands on the remote machines. Similarly to~\cite{blockbench} we
distinguish between the roles \emph{miner} and \emph{client}. The nodes of the
former type are accountable to generate new blocks while the nodes of the latter
type create and propagate transactions and both verify the blocks\footnote{To
reduce the number of test variables we consider only full nodes.}. We can assign
multiple roles to a single machine. In this case we run \emph{one distinct geth
instance} for each different role. The coordinator copies the right genesis
block in the test machines.

In each run of test we distinguish at least two phases:
\begin{enumerate*}
  \item the setup and
  \item the test itself.
\end{enumerate*}

\paragraph{Setup}
The setup consists in iterating on the test machines twice:
\begin{itemize}
  \item in the first iteration the required ethash data structures are generated,
  \item in the second one a genesis file is used to initialize the ethereum world
  state.
\end{itemize}


\paragraph{Ethash data structure}
During the setup phase the nodes with the roles miner generate the DAG and the
cache for the first two epochs, while the clients generate only the caches,
because they are required only to validate blocks. We generate the DAGs for the
first two epochs, because ethash uses double buffer of DAGs to grant a smooth
switch between epochs~\cite{bib:dagger-hashimoto}.


\paragraph{Genesis file}
The genesis file contains useful information to create (deterministically) the
genesis block and the initial state. We report an extract of the genesis file
we used in~\autoref{listing:genesis}. It is simply JSON file, which specifies
several parameters. Most of them specify directly the attribute of the block
number $0$, which are shown in \autoref{fig:world-state}. Here we describe
only the fields that are not directly a parameter specification:
\begin{itemize}
    \item \texttt{config}: It describes the network id and the number and hash
    of blocks that marks the entry into force of the Ethereum Improvement 
    Proposals (EIP), which indicate incompatible changes in the protocol or
    simply a new version of Ethereum
    \item \texttt{alloc}: It specifies an initial allocation of Ether for the
    accounts\footnote{This possibility has been exploited for the so-called
    Initial Coin Offering (ICO) used by the Ethereum Foundation to obtain fiat 
    currency to finance the project.} we use in the tests.
\end{itemize}
Obviously, each node of the network should be initialized with the same genesis 
file. Otherwise, the hash of the genesis block differs and the peers cannot
establish a connection with the Ethereum Wire Protocol, as described
in~\autoref{sec:ethereum-wire-protocol}. Thus, the genesis file for the main 
network  and the official Ethereum test networks are hard-coded. To create a 
new private Ethereum network it is sufficient to use a new genesis file, in 
which some parameters are changed. Therefore, in our genesis file we use an 
arbitrary network id and nonce, so that packets of different networks are 
simply dropped.

\begin{figure}[H]
    \lstinputlisting[caption={An extract of the genesis file used in the tests.}, label=listing:genesis]{./res/code/genesis-file.json}
\end{figure}


Apart from the arbitrary values, that is the id and the nonce, and the data
that perhaps are required by the system do not have influence on the transaction
throughput, some parameters require some justification, that we provide in the 
remainder of this paragraph.

The \textbf{timestamp} of the genesis file, that corresponds to the one of the
genesis block, influences the difficulty of the first blocks, and transitively
of all the blocks. Since, for time constraints, we want to run each test for few
minutes we want to avoid sudden decreases in the difficulty due to a too old
timestamp. Therefore, to prevent these unwanted changes in the difficulty value,
during the second loop of the setup phase, the initiator reads its timestamp and
gives it as parameter to the peers\footnote{Before starting the test all the
peers should be configured, therefore using the timestamp of the coordinator
does not assume fine-grained coordinated clocks.}.

As already described in \autoref{sec:consensus:algorithm}, the
\textbf{difficulty} is an adaptive parameter that determines how much effort
should be invested in the creation of a new block. In our tests we used the same
hardware and same operating system in all nodes and for all miners (we
deliberately used only one thread for mining). Therefore to find a suitable
start value for the difficulty for the different configurations we ran a
simulation with one, two, four and eight miners for 24 hours.
\autoref{fig:start_difficulty_raw} shows how the difficulty changed with the
different number of miners. We can notice that in our homogeneous system the
final values of difficulties have a quasi linear dependency with the number of
miners. For the test we took as initial value the median of the last $100$
blocks. These values are represented in~\autoref{table:start-difficulty}. We
reported also the coefficient of variation to show that the values of the last
$100$ blocks are pretty stable.
\begin{figure}
  \begin{center}
    \includegraphics[width=0.8\textwidth]{./res/img/start_difficulty_all.png}
    \caption{The growth of the difficulty in the 24 hour run.}
    \label{fig:start_difficulty_raw}
  \end{center}
\end{figure}

\begin{table}
    \begin{center}
        \begin{tabular}{c | c | c}
            Number of Miners & Difficulty & Coefficient of variation \\
            \hline
            1 &  404559 & 0.0073\% \\
            2 &  821994 & 0.2330\%\\
            4 & 1711150 & 0.1458\%\\
            8 & 3409299 & 0.1742\%\\
        \end{tabular}
        \caption{The median of the last $100$ blocks, value used in the tests as
            start difficulty and the coefficient of variations.}
        \label{table:start-difficulty}
    \end{center}
\end{table}


The \textbf{gas limit} in the genesis block determines the gas limit of the
first block. The gas limit of the subsequent blocks can be determined freely by
the miners but must be contained in a range obtained by summing and subtracting
to the gas limit a portion of itself~\cite{wood2018ethereum}. Thus, in a
relative brief simulation the initial value is fundamental. Due to the
considerations of \autoref{sec:background}, we decided to use a large value for
the gas limit, i.e. $500 \cdot 21000$, that corresponds to the gas to execute
$500$ transactions without execution of code\footnote{To give a term of
reference, we can consider that currently the gas limit for the main Etherum
network is circa $8$ Milion, that corresponds roughly to $381 \cdot 21000$.}.



\paragraph{Concrete Configuration}
We measure the throughput with different number of miners ($1$, $2$, $4$ and 
$8$), fixing the number of clients at $16$. The miner machine executes only one
geth instance, while the client machines executes two instances.

For our tests we used the scaleway
platform\footnote{\url{https://www.scaleway.com/}}. We rent $17$ (one master
and $16$ slaves) START1-S servers,
which dispose of $2$ X86 64 bit Cores and $2$ GB of RAM and an internal
bandwidth of $1$ Gbit/s, so that the network would not be a bottleneck. Each
server runs Ubuntu 16.04 and has installed, \texttt{geth} version 1.8.11-stable.
During the tests the master acts also as bootstrap node. We described its role
in \autoref{sec:network-layer}.

More details about the implementation and the configuration can be obtained
by inspecting the repository, containing the code for the 
tests~\footnote{\url{https://github.com/gfornari/ethereum-test/tree/benchmark}}.


\subsubsection{Results}

\paragraph{Maximal throughput}
\label{sec:max-troughput}
We measure the maximal throughput with different number of miners, fixing the
number of clients at $16$ (2 clients per machine) and the number of emitted
transactions constant at one every $50$ ms. \todo{Specify params like
difficulty, gas limit, execution time}

\begin{table}[h]
  \centering
  \begin{tabular}{lcccc}
    \hline
    & 1 miner & 2 miners & 4 miners & 8 miners \\ \hline
    Avg throughput & 8.01 & 7.17 & 5.83 & 6.65 \\ \hline
    Avg blocks & 24.00 & 37.80 & 26.60 & 30.00 \\ \hline
  \end{tabular}
  \caption{Average throughput and number of mined blocks on 5 runs.}
  \label{tab:max-troughput}
\end{table}


\paragraph{Maximal throughput with high gas limit}
\label{sec:max-throughput-high-gaslimit}
As before, we measure the maximal throughput with different number of miners,
fixing the number of clients at $16$ (2 clients per machine) and the number of
emitted transactions constant at one every $50$ ms, but with a high block gas
limit to avoid its limitations including pending transactions in the next block.
\todo{Specify params like difficulty, gas limit, execution time}

\begin{table}[h]
  \centering
  \begin{tabular}{lcccc}
    \hline
    & 1 miner & 2 miners & 4 miners & 8 miners \\ \hline
    Avg throughput & 8.17 & 11.55 & 9.02 & 11.07 \\ \hline
    Avg blocks & 33.60 & 27.20 & 37.20 & 29.80 \\ \hline
  \end{tabular}
  \caption{Average throughput and number of mined blocks on 5 runs with high gas limit.}
  \label{tab:max-troughput-high-gaslimit}
\end{table}

