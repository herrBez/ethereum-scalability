\subsubsection{State}
\label{sec:world-state}

The \textbf{World State}, referred as the \emph{state}, in its simplest
definition, is a mapping between account addresses and account states.

\begin{figure}[h]
  \centering
  \includegraphics[width=\textwidth]{./res/img/world-state.pdf}
  \captionsource{Representation of the World State.}{Adapted from \url{https://ethereum.stackexchange.com/a/757}}
\label{fig:world-state}
\end{figure}

\autoref{fig:world-state} represents it. The block header contains 15 fields,
among which the \verb+stateRoot+, the \verb+transactionsRoot+ and the
\verb+receiptsRoot+. All these three fields are a hash\footnote{Here we intend
the Keccak 256-bit.} of the root of a Merkle Patricia tree data structure:

\begin{itemize}
  \item the \verb+stateRoot+ represents the state tree, after all transactions
  are executed and finalisations applied (for a complete specification of the
  fields, refer to \cite{wood2018ethereum}), storing the mapping between account
  state and account address
  \item the \verb+transactionsRoot+ represents the list of the transactions included in
  the block
  \item the \verb+receiptsRoot+ represents the receipts list of the transactions
  included in the block, which shows the \emph{effect} of each transaction. In
  \cite{merklingethereum}, the Ethereum's founder, Vitalik Buterin,
  writes that with the receipt information, someone can answer queries like
  ``Tell me all instances of an event of type X (e.g. a crowdfunding contract
  reaching its goal) emitted by this address in the past 30 days''.
\end{itemize}

Some other block header fields are described in \autoref{sec:tests}. For a
complete fields list and description, we refer to the Ethereum protocol
specification~\cite{wood2018ethereum}.

\paragraph{Merkle Patricia tree} A Merkle Patricia tree is a data structure
which derives from a radix tree reducing the space complexity
\cite{patriciatree}. It has a single root, each node is the hash of its children
and the leaves are the actual data, that is, for the case of the
\verb+stateRoot+, the accounts states, which in turn include the
\verb+storageRoot+, a hash of another Merkle Patricia tree representing the
storage contents of the account state.

Since each node is the hash of its children, if any data in the tree change,
recursively and correspondingly all the ancestors nodes have to change from the
changed node to the root node. This property allows us to uniquely identify a
tree having just the root node. This is worth to notice because it allows the
nodes of the network to verify that big data-structures (like the World State)
correspond to the one contained in the blocks' header by simply comparing a
$256$-bit long hash. Moreover, this feature can be exploited to create the light
nodes, as described in~\autoref{sec:node-types}.


\subsubsection{Accounts}
\label{sec:accounts}

The accounts are also called the \emph{state objects} and are essential for the
user to interact with the Ethereum blockchain via transactions.

There are two types of accounts:

\begin{itemize}
  \item the \textbf{non-contract account} (referred to as \emph{account} or
  Externally Owned Account, EOA)
  \item the \textbf{contract account} (referred to as \emph{contract}), which
  has EVM Code associated with it and is controlled by its contract code
\end{itemize}

A \emph{non-contract account} has no EVM Code associated with it and is
controlled by a private key. This type of account can send a message to another
non-contract account, that is a value transfer, or to a contract account in
order to trigger the execution of code. The state of an account is essentially
its balance.

A \emph{contract account} has EVM Code associated with it and is controlled by
it. This type of account cannot send messages or transactions on its own, but
only as a response to a trigger. The state of a contract is its balance and its
contract storage. A contract code is executed by the EVM, can manipulate its own
persistent storage and can send internal transactions (i.e., message calls) to
other contracts.

Both the types of account have an associated \emph{nonce}. In the case of an
externally owned account this value corresponds to the number of transactions
sent by the account while in the case of a contract account it corresponds to
the number of contract creations performed by the account. Obviously, this value
is always positive and can only be increased.

When creating a transaction, the externally owned account should specify its
nonce. This guarantees that the order of transactions of a single account are
processed in the order specified by the sender. Without this expedient something
unforeseen can happen, for example, the balance of the account can get under a
certain threshold and so other transactions cannot be performed. Since contracts
cannot perform transactions but can still create other contracts once invoked,
the nonce is used to guarantee that each different created account have a
different contract address. Indeed, the contract address is obtained as function
of the address of the creator and its nonce~\cite{wood2018ethereum}.
