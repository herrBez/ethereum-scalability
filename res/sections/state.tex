\subsection{State}

We can distinguish two different states in the Ethereum system:

\begin{itemize}
  \item the \textbf{World State}, referred simply as the \emph{state}, which is
  a mapping between account addresses and account states
  \item the \textbf{Machine State}, the state of an Ethereum Virtual Machine
  (EVM) (see \autoref{sec:evm})
\end{itemize}

\begin{figure}[h]
  \centering
  \includegraphics[width=\textwidth]{./res/img/world-state.pdf}
  \captionsource{Representation of the World State.}{Adapted from \url{https://ethereum.stackexchange.com/a/757}}
\label{fig:world-state}
\end{figure}

\autoref{fig:world-state} represents the Ethereum World State. The block header
contains 15 fields, among which the \verb+stateRoot+, that is the
hash\footnote{Here we intend the Keccak 256-bit.} of the root node of the state
trie, after all transactions are executed and finalisations applied (for a
complete specification of the fields, refer to \cite{wood2018ethereum}). This
corresponds to the root hash of a Merkle Patricia tree data structure which
store the mapping between account state and account address

A Merkle Patricia tree is a data structure which derives from a radix tree
reducing the space complexity \cite{patriciatree}. It has a single root, each
node is the hash of its children and the leafs are the actual data, that is the
accounts states, which in turn include the \verb+storageRoot+, a hash of another
Merkle Patricia tree representing the storage contents of the account state.

Since each node is the hash of its children, if any data in the tree change,
recursively and correspondingly have to change all the ancestors nodes from the
changed node to the root node. This property allows us to identify uniquely a
tree having just the root node. This is worth notice because it influences
the system scalability. Briefly, we can distinguish between \emph{light nodes}
and \emph{full nodes}. The formers store just the blocks' headers, while the
latters store the entire blockchain. The tree's property we just mentioned
allows the light nodes to the validation process even without storing all the
data. This is better analyzed in \autoref{sec:scalability} talking about the
scalability.

\subsubsection{Accounts}
\label{sec:accounts}

The accounts are also called the \emph{state objects} and are essential for the
user to interact with the Ethereum blockchain via transactions.

There are two types of accounts:

\begin{itemize}
  \item the \textbf{non-contract account} (referred to as \emph{account} or
  Externally Owned Account, EOA)
  \item the \textbf{contract account} (referred to as \emph{contract}), which
  has EVM Code associated with it and is controlled by its contract code
\end{itemize}

A \emph{non-contract account} has no EVM Code associated with it and is
controlled by a private key. This type of account can send a message to another
non-contract account, that is a value transfer, or to a contract account in
order to trigger the execution of code. The state of an account is its balance.

A \emph{contract account} has EVM Code associated with it and is controlled by
it. This type of account cannot send messages or transactions on its own, but
only as a response to a trigger. The state of a contract is its balance and its
contract storage. A contract code is executed by the EVM, can manipulate its own
persistent storage and can send internal transactions (i.e. message calls) to
other contracts.
