\section*{Introduction}

The purpose of this report is to investigate the architecture and the 
scalability of Ethereum~\cite{wood2018ethereum}. Ethereum is based on the 
Blockchain technology. This technology is one of the most active research
fields in ICT in the last couple of years, despite the first appearance of
blockchain in the actual form is due to Satoshi Nakamoto's groundbreaking 
paper, ``Bitcoin: A peer-to-peer electronic cash system''
(2008)~\cite{bib:bitcoin}. 

The aim of this technology is to provide a total order of the 
transactions in a distributed ledger without relying on a trusted third party, 
e.g. a bank~\cite{bib:the-quest}. Not relying on a trusted central authority
may lead to practical issues like transaction repudiation and the infamous 
\textbf{double-spending problem}. The former is self-explanatory and can
be solved by digital signatures, while the latter consists in using the same
digital token to pay multiple entities and remained unsolved until bitcoin 
appeared. 

The idea to solve the double spending problem consists in making all the peers
aware of all transactions and to disallow rewriting of history. This implies
that the information stored on the blockchain is immutable~\footnote{After a 
confirmation time as explained in~\autoref{sec:consensus}} and tamper-proof,
which are desirable properties in a public ledger.