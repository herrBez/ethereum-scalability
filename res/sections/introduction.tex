\section*{Introduction}

The purpose of this report is to investigate the architecture and the 
scalability of Ethereum~\cite{wood2018ethereum}. Ethereum is based on the 
Blockchain technology. This technology is one of the most active research
fields in ICT in the last couple of years, despite the first appearance of
blockchain in the actual form is due to Satoshi Nakamoto's groundbreaking 
paper, ``Bitcoin: A peer-to-peer electronic cash system''
(2008)~\cite{bib:bitcoin}. 

The aim of the blockchain technology is to provide a total order of the 
transactions in a distributed ledger without relying on a trusted third party, 
e.g. a bank~\cite{bib:the-quest}. Not relying on a trusted central authority
may lead to practical issues like transaction repudiation and the infamous 
\textbf{double-spending problem}. The former is self-explanatory and can
be solved by digital signatures, while the latter consists in using the same
digital token to pay multiple entities and remained unsolved until the 
appearance of Bitcoin.

This technology has found, apart from merely financial applications, other
applications such as auctions and supply chain.


Bitcoin is a state transition system, in which there is a transition from
a valid state to another valid state through a valid \emph{transaction}. The
state consists in the balance of the addresses\footnote{The addresses 
correspond to a private/public key pair. Furthermore, each peer of the 
network can have zero or more addresses}. Each node in the network maintains a
local copy of the state and updates its \emph{own} copy of the state in a 
deterministic way according to the transactions. Therefore, to have an exact 
replica in each node, the order of transactions should be total and agreed by 
every member of the network. The mechanism through which this total order is
provided and maintained is the blockchain, which is literally a chain of
blocks. Each block of the chain contains an ordered list of transactions and is 
connected to the previous block by inserting the hash of the previous block in 
its header. Each node of the network has the faculty to create transactions and
must sign them to show that it is in possession of the private key corresponding
to a given address. The transactions are spread in the network through
gossip protocols. Once a node receives a new transaction, it verifies 
the validity of the transaction and if the checks pass the node sends the 
transactions to the peers, it knows. Eventually, the transactions are received
by a member of the network, who tries to find a nonce such that the hash of the 
block is smaller than a given value. Since this task is computationally 
expensive, if the node successfully find this value, it adds at the beginning 
of the transaction list a transaction in which it assign to a beneficiary 
address an amount of newly minted coins, according to the protocol's 
rule\footnote{This value was initially $50$ Bitcoins. This reward halves every 
$210000$ blocks. Currently it is 12.5 Bitcoins. Around year $2140$ no coins 
would be minted~\cite{bib:masteringbitcoin}.}. In addition to this reward, it 
receives also fees from the sender of the included transactions. The members of
the network, who try to create new blocks, are called miners, because their 
action resembles the extraction of precious metals. The miners are incentivized 
to create valid blocks, that is, it contains valid transaction and the solution 
to the puzzle  is correct, because the other peer of the network accept 
\emph{only} valid blocks. They can verify the correctness of the transactions, 
because they have a local copy of the state and can verify the correctness of 
the nonce by simply computing a single hash. It is worth notice that multiple
parties try to create new blocks concurrently and therefore it is possible that 
multiple version of the blockchain co-exist. So, a mechanism to select the 
canonical blockchain, i.e. the valid one is needed. In the case of bitcoin
it is simply the longest chain, because it correspond to the one with more work
invested on it. The co-existance of multiple blockchain can be very useful
in case of a network partition, indeed, once the partition is over the 
peers can agree on the blockchain. The drawback of this system is that
there is no consensus finality~\cite{bib:the-quest}, thus it is necessary to
wait a certain number of blocks (confirmation blocks) to be sure that the 
transactions are really confirmed. The number of confirmation blocks in bitcoin
is six, which correspond to circa one hour.

The double spending problem is solved by letting each peer of the network
know the current state and the transactions that are already spent.


Apart Bitcoin a lot of cryptocurrencies, also known as 
\emph{altcoins}\footnote{The contraction of alternative coins}, came out.
They have different peculiarities but the general idea is the same as bitcoin.
In the literature, it is common to distinct between permissionless and 
permissioned blockchain. The former are blockchains in which everyone who has 
an Internet connection and a client can participate while the latter requires 
authentication and is commonly used by banks or consortium of companies. 
Prominent examples of permissionless blockchains are Bitcoin and Ethereum, 
while a representative permissioned blockchain is hyperledger.







