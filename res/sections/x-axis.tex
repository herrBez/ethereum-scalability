\subsubsection{X-Axis: Horizontal Duplication}

The x-axis of the cube of the scalability is concerned with the horizontal
duplication and cloning with absolutely no bias, running each identical copy of
the system on a different server. Usually, the work is distributed by a load
balancer.

Reasoning on the x-axis is typically easy and the implementation can be fast,
but the data sets have to be replicated in their entirety which increases
operational costs.

% descrivere un esempio virtuoso fuori dall'architettura di Ethereum
In order to better understand the concept, we bring an example of a common
architecture which scales on this axis. Suppose you are running your own
e-commerce startup of wine. The business is going great and suddenly you have to
face the explosive growth of HTTP requests to your server. Since you have an
early stage startup, so far you had only a single server running on a single
machine, but you know that, if everything goes as hoped, you cannot scale-up for
a long time. So, you decide to take you Web server codebase and deploy an
identical copy of it. Right after, you set up
nginx\footnote{\url{http://nginx.org/}} as HTTP load balancer. The two Web
servers now work in parallel and access the same database, thanks to your
ability to write \emph{stateless} servers. The statelessness is an important
property in this scenario, it avoids dependency between requests, that is
the server can process a request without needing to access the information of
another one. In the example, if you would not have written a stateless server
(i.e. stateful), one request arrived at one of the two servers could be the ones
on which another request being processing on the other server depends on, hence
without permitting to successfully fulfill the latter.

\paragraph{State}
Let's clarify the concept of \emph{state} in order to better understand the
importance of it scaling on the x-axis. We said that, in other words, an
application that uses state chooses the next action to be performed evaluating
the current execution condition \cite{bib:art-of-scalability}. This definition
holds for the protocols as well. A common example of stateless protocol is HTTP,
since it does not need to know anything about a previous request having all the
information needed to fulfill the current request. On the contrary, an example
of stateful application is a possible implementation of user session (which can
be done also with a stateless approach), in which a user is authorized to
request some resources only after an authentication request. In this setting,
the result of the authentication could be stored in the server making it
stateful, i.e. some requests are dependent on the authentication request.

% discutere sui mining pool

% discutere su Infura