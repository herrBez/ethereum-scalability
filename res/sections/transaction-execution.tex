\subsubsection{Transaction Execution}
\label{sec:tx-execution}

The transaction execution is the mechanism through which the world state is
updated. It represents a transition from one valid state to another valid state.

A transaction specifies a \emph{receiver} and the \emph{value} that must be
transferred from the sender to the receiver. Moreover, to avoid the abuse of the
resources (CPU and storage) of the full nodes forming the network and to ensure
that all executions terminate, the concept of \emph{gas} is introduced. In this
execution model each action that must be performed by the members of the network
has an associated cost expressed in gas. In particular, each EVM byte-code
instruction, increase in the storage space by a contract and the transaction
itself have a fixed associated cost. A transaction specifies its \emph{gas
price} and its \emph{gas limit}. The former is the price of a unit of gas and is
bound to a particular execution. The higher this price the higher the
possibility that the miner will include this transaction in the blockchain.
Usually the miners advertise the minimum gas price they are willing to accept.
The latter is the maximum amount of gas the executor is ready to consume for
this particular execution.

\begin{figure}[h!]
	\begin{center}
		\includegraphics[width=0.60\textwidth]{./res/img/transaction-execution.pdf}
	\end{center}
	\caption{The steps of the transaction execution algorithm.}
	\label{fig:tx:execution}
\end{figure}

\autoref{fig:tx:execution} summarizes the steps of the algorithm for the
transaction execution:
\begin{enumerate}
	\item The transaction has to pass some simple validity checks, e.g.\ the
	transaction should be a well-formed RLP encoded string and the initiator of
	the transaction should have a balance big enough to afford the transaction.
	Moreover, since a transaction should be included in a block and the block
	has in turn its own gas limit, it should be also true that the sum of the
	accumulated gas used by the already included transactions and the gas limit
	of this transaction are smaller than the block's gas limit. If these checks
	fail, the transaction is simply not included in the block in case of miner
	or it indicates that the block is invalid in case of verifier.
	\item The nonce of the initiator of the transaction is incremented by one
    and its balance is reduced by the product of the gas limit and the gas
    price. This modification to the state is irreversible.
	\item Now, depending on whether the receiver address is given or not we
	should make a distinction between:
	\begin{enumerate}[label=\alph*.]
		\item Message Calls (\autoref{sec:message-call})
		\item Contract Creation (\autoref{sec:contract-creation}).
	\end{enumerate}
	During the execution of the message call or contract creation the system
	keeps track of the \emph{transaction substate}, i.e., some important
	information that are later used to complete the state transition. The
	transaction substate includes the \emph{touched accounts}, the set of
	accounts that will be discarded following the completion
	(\emph{self-destruct set}) and the \emph{refund balance}, that is an amount
	of gas that is incremented by removing elements from the world state, e.g.,
	by setting a non-zero value to a zero value in the storage or by removing a
	contract from the state.
	\item Once the message call or the contract creation are concluded, the sum
    of the remaining gas and the refund balance are refunded to the initiator of
    the transaction at the transaction's gas price\footnote{This sum cannot
    exceed the initial allocated price~\cite{wood2018ethereum}, in other words
    the refund balance can be used to mitigate the transaction cost, but not to
    profit.}. If the message call or contract creation did not successfully
    complete, the state is reverted and the refund balance is zeroed, since its
    modifications to the state are not considered. The gas used is given to the
    beneficiary address (i.e. the miner) who built and finalized the block.
    Finally, the self-destruct set and the touched accounts that became empty or
    dead after the transaction should be deleted from the world state.
\end{enumerate}
