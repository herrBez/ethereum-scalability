\section{Conclusions}
\label{sec:conclusions}
The Blockchain technology provides a way to find a total order of transactions
in a distributed system without relying on a trusted third party. This permits
to have an exact copy of the state in each node of the network, because each
node starts from the same state and changes the state according to the
transactions, whose execution is deterministic. The main advantage of this
technology with respect to traditional transition systems, is its decentralized
nature.

In this report we took into consideration Ethereum, a representative of
permissionless blockchain technology. This system is particularly interesting
because it supports a general-purpose execution environment, the EVM.

In the first part of this work we proposed a decomposition of the Ethereum's
architecture in logical layers. The five stacked layers abstract from the
implementation details and offer a conceptual organization useful to compare
different blockchain proposals, although permitting flexibility as the case of
the EVM which is a cross layer. Moreover, this decomposition allows a
component-oriented development which helps reasoning on the system.

The focus of the second part is the analysis of the scalability of Ethereum. To
do so, we analyzed the existing literature (\autoref{sec:background}) and
confirmed empirically that the current version of Ethereum reaches a maximal
transaction throughput lower than two dozens per second, even if the
transactions do not require computations. This value is not influenced by the
number of miners, but rather by the block size and the block interval.
Furthermore, we analyzed the motivation of this reduced transaction throughput
with the aid of the cube of scalability~\cite{bib:art-of-scalability}. We argue
that the current version of Ethereum is not developed in any direction of the
cube by analyzing the current specification. Thereafter, we categorize the
scalability improvement proposals based on the axes they will affect. From this
analysis, it is clear that the most efforts of the community are concentrated
on the z-axis of the cube~\autoref{sec:z-axis} with proposals like Plasma and
sharding. Sharding is a scaling strategy already seen in the database systems,
that is, systems which have to manage persistent data. Maybe, this common duty
between Ethereum and database systems leads the z-axis of the Scale Cube.
