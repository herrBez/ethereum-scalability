\label{sec:consensus}

The ultimate aim of the blockchain technology is to provide a \textbf{total
order to transactions} in a distributed ledger~\cite{bib:the-quest} without
relying on a trusted third party. This permits to solve the double spending
problem~\cite{bib:bitcoin}. Moreover, in Ethereum the order of transactions can
also affect the execution of smart contracts by altering the content of the
storage.

Finding an agreement on the order of transaction (i.e. the actual blockchain)
and the world status is crucial, thus multiple consensus algorithms were
proposed~\cite{}. Ethereum follows an idea very close to the consensus algorithm
of Bitcoin, which is also known in the literature as \textbf{Nakamoto
consensus}~\cite{}.

The basic idea of this algorithm consists in:
% enumerate* means in-line enumeration
\begin{enumerate*}[label=(\arabic*)]
	\item accepting only valid blocks with regards to some validation criterion,
	and
	\item relying on a \textbf{selection rule} to choose between two different
	valid forks, depending on the amount of work performed in each fork.
\end{enumerate*}

The \emph{validation criterion} used to determine whether a block is valid or
not consists in
\begin{enumerate}
	\item checking that the blocks and transactions are well-formed
	\item re-performing all the transactions
	\item re-executing \emph{all}	the EVM computations to verify whether the
	transaction receipts and the state root contained in the propagated block
	(\autoref{fig:world-state}) are valid.
\end{enumerate}

The \emph{selection rule} is required to avoid the infamous double spending
problem. Indeed, in Bitcoin (and as well in Ethereum) the assumption is that the
majority of computing power belongs to good players who will follow the
rules\todo{add the exact proportion}. Therefore, in order to prevent bad agents
to rewrite the transactions history with a high probability~\cite{bib:bitcoin}
the issuer of new block should prove that he invested resources in its creation
by solving a computational heavy task. This mechanism is known as
\textbf{Proof-of-Work} (PoW). In Bitcoin the selection rule consists in
accepting the longest chain that corresponds roughly to the one with more work
invested on it. In Ethereum, instead, the selection rule is a simplified version
of the Greedy Heaviest-Observed Sub-Tree (GHOST) selection
rule~\cite{wood2018ethereum}. The main idea of this rule is to take into
consideration not only the longest chain, but also the blocks that are not
included in the chain (i.e. the ommers) to obtain a more secure and scalable
system~\cite{bib:ghost}\todo{Add explanation}.

%In the case of bitcoin this task consists in finding a number (\emph{nonce})
%such that the hash of the block along with the nonce itself begins with a
%number of leading zeros. The difficulty of the task increase exponentially with
%the number of leading zeros. Since the output of the hash function is not
%predictable, the creator of a new block should try each possible nonce in a
%brute-force manner. Once a valid nonce is found the block is spread and the
%receiver should only perform a single hash to check that the result is valid.

