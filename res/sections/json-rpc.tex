\subsection{External Interaction}

So far we described how Ethereum \emph{internally} works but did not provide
a description of \emph{how} a user or an external application can interact with 
the system, e.g. how transactions are sent by users or how can an application
read the balance of a given address.

To this extent the Ethereum community has developed a JSON RPC
API~\cite{bib:json-rpc} that is compliant with the
JSON RPC 2.0 specification~\cite{bib:json2012json}.
JSON RPC is stateless and can be used on top of diverse protocols (e.g. http).
It allows external actors to invoke the exposed API methods by sending
JSON encoded requests. These should specify the
API version, the API method, the parameters encoded as a list and a nonce
that binds a request to a reply.
For the sake of clarity we show an example that calls the method 
\verb|eth_blockNumber|, that returns the number of blocks:

\begin{lstlisting}[language=bash]
curl -X POST -H "Content-Type: application/json" --data \
'{"jsonrpc":"2.0","method":"eth_blockNumber","params":[],"id":1}' \
http://localhost:8545
\end{lstlisting} 
The server replies with a JSON string that contains the result and the same
nonce as the request.




In addition to the JSON RPC API, a Javascript API was developed. It is provided
as a Javascript library, 
\texttt{web3}\footnote{\url{https://github.com/ethereum/web3.js}}, that allows
Javascript code to
communicate with a running Ethereum client. It is simply a convenient
Javascript wrapper for the JSON RPC calls~\cite{bib:javascript-api}.

For a complete list of the methods supported by the two APIs we refer to the
respective documentations~\cite{bib:json-rpc, bib:javascript-api}. 

