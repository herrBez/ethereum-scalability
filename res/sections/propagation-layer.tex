\subsection{Propagation Layer}
\label{sec:propagation-layer}

The propagation layer objective is to spread the information regarding the
blockchain among all the peers. It exploits the contact information obtained
in the Network Layer.

In this layer, we can identify three main components used to achieve the
objective:
\begin{itemize}
  \item \emph{RLP}, the main encoding method used to serialize objects in Ethereum. In
  this layer, it is used as a marshaling algorithm
  \item the RLPx protocol, intended as the transport
  protocol\footnote{\url{https://github.com/ethereum/devp2p/blob/master/rlpx.md\#transport}},
  which plays a role very similar to the OSI transport layer. For the sake of
  disambiguation, we refer to it as \emph{RLPx Transport Protocol}
  \item the \emph{Ethereum Wire Protocol}, useful to spread the blockchain
  information.
\end{itemize}


\subsubsection{RLPx Transport Protocol}
\label{sec:rlpx-transport-protocol}

The aim of this protocol is to provide a transport layer for the messages. In
essence, it is an encrypted and authenticated TCP with payload encoded with the
Recursive Length Prefix (\textbf{RLP}) marshaling algorithm. Two peers who wants
to communicate with each other with RLPx should as a first step perform a
handshake, in which they exchange a cryptographic secret\footnote{It is beyond
the scope of this report to describe the exact procedure. For further details we
refer to the official documentation \cite{rlpx} and to the go ethereum
implementation \path{}.}\todo{Add reference to implementation (see footnote)},
that is used to encrypt and authenticate the subsequent RLPx messages between
them.


\subsubsection{Ethereum Wire Protocol}
\label{sec:ethereum-wire-protocol}

The Ethereum Wire Protocol is an application level sub-protocol of the \devpp{}
Wire Protocol. With the term sub-protocol, it is intended an extension of the
base protocol defining more messages which are added to the base ones.

The base \devpp{} Wire Protocol exploits RLPx Transport Protocol, that is, the
\devpp{} Wire Protocol's messages are sent through the RLPx Protocol in its
payload. \devpp{} Wire Protocol has four message types: \verb+Hello+,
\verb+Disconnect+, \verb+Ping+ and \verb+Pong+.

The \verb+Hello+ (Handshake) message is used upon connection and upon receiving
a message of this kind. This message specifies, among others, the protocol
version, the \textbf{capabilities} and their version, the port on which the
client listens and the node ID. The capabilities are the application level
protocols supported by the sender of the Hello Message. At the moment of
writing, only the \emph{eth} (Ethereum), \emph{les} (Light Ethereum
Subprotocol), \emph{bzz} (Swarm) and \emph{shh} (Whisper) protocols are used.
The \verb+Disconnect+ message notifies the receiver that the sender is going to
disconnect itself. The disconnect message can specify optionally an integer that
encodes a reason. For the complete reason codes, we refer to the \devpp{}
specification~\cite{devp2pwire}. The \verb+Ping+ and \verb+Pong+ message are
used to check whether the counterpart is still on-line or not.

% TODO fix onwards

\paragraph{Ethereum Wire Protocol}
The Ethereum wire protocol (\emph{eth})~\cite{bib:ethereumwireprotocol} is used
to spread the information about the blockchain and for the synchronization.
There are several versions of this protocol. Throughout this section we will
consider only the versions $62$ and $63$ (which are compatible) that are
currently supported by the go Ethereum client (v. 1.8.3).

The first message that must be exchanged between two peers
is the \textbf{Status} message. This kind of message is used to exchange,
among others, the protocol version, the network IDs, the total difficulty of the
heaviest chain known, the hash of the best known block and
finally the genesis block's hash. This message should be sent only during
the handshake phase.
If the network id or the genesis block's hash do not match or the supported
\emph{eth} protocol versions are not compatible, the peers should drop
the connection, since they are either on different chains or are not able to
communicate with each other.




\subparagraph{Protocol Version 62 - Model Syncing}
To spread the presence of one or more blocks to peers that are not aware
of them the \textbf{NewBlockHashes} message type is used.
Moreover the \textbf{Transactions} message type spreads transactions to
peers who are not aware of them. It is specified that
in the same session a peer should not send twice the same
transaction to a
recipient\footnote{To this extent the
geth implementation (in the file \path{eth/peer.go}) keeps track for each
peer of the set of transactions hash (\texttt{knowTxs}) and the set of
block hash (\texttt{knownBlocks}) known to be known by it.}.
The \textbf{GetBlockHeaders} message type requests to the recipient at most
\texttt{maxHeaders} block headers descendant from the block with a given
number or a given hash.
The recipient of the message should respond with a \textbf{BlockHeaders}
message, in which it has the faculty to send a reply with less than
\texttt{maxHeaders} headers. Clearly, if the recipient of a message
is not aware of any descendant of the given block, it sends a valid empty
reply. Furthermore to request and to receive the real content of the blocks the
peers have the \textbf{GetBlockBodies} and \textbf{Blockbodies} messages.
The requester specifies the hashes of the blocks it wants and the recipient
replies with the bodies (i.e. the transactions and the uncles) of
the required block(s).
Finally, the \textbf{NewBlock} message spreads a single new block.

\subparagraph{Version 63 - Fast synchronization}
From version v. x of geth it is possible to perform a fast synchronization.
This synchronization type does not require that a node performs \emph{all} the
computations happened during the history (i.e. the whole EVM instructions).
Indeed the synchronizer downloads along the blockchain the transaction receipts,
i.e. useful information about the execution of the transaction.
This allows the synchronizer to deal only with the verification of the
proof-of-work. At least in geth this synchronization is possible only by the
first synchronization for security
reasons~\footnote{\url{https://github.com/ethereum/go-ethereum/pull/1889}}.
After the synchronizer reaches a \textit{pivot point} (last block $- 1024$) it
retrieves the whole current state and afterwards processes the blockchain
normally.
To perform this synchronization the clients have at their disposal
the \textbf{GetReceipts} and \textbf{Receipts} messages, which are the
request for the receipts given the hash and the replies, respectively.
Besides these there are also the \textbf{GetNodeData} and
\textbf{NodeData} message types which provides the mean to query and
send the required version of the state. The former message types take as
input a variable number of hashes and the latter responds with the content.




