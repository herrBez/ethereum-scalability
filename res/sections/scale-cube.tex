\subsection{The Scale Cube}
\label{sec:scale-cube}

The \emph{Scale Cube}, as shown in \autoref{fig:scale-cube}, uses the
representation of a cube drawn on a 3-dimensional Cartesian space to define
three different scaling directions an architecture can develop in order to
grow and shrink along with the demand. Although in a Cartesian space we could
measure the cube size, the Scale Cube does not provide actual metrics to
quantify the scalability, but rather a way of thinking about scale; that is what we
mean with \emph{scaling directions}.

\begin{figure}[h]
	\begin{center}
		\includegraphics[width=0.8\textwidth]{./res/img/scale-cube.pdf}
	\end{center}
	\caption{The Scale Cube.}
	\label{fig:scale-cube}
\end{figure}

The use of one or two axes does not preclude the possibility to scale on the
third axis. The initial point with coordinates $(0,0,0)$ means least
scalability. The prototypical system at the initial point consists of a single
monolithic application and storage retrieval system likely running on a single
physical machine~\cite{bib:art-of-scalability}. Of course, it might scale up,
that is it could run on a more powerful machine, but it won't scale out, hence
it will not take advantage of a distributed architecture. All of the three axes
scales well from a transaction perspective, that is, in our case, the
transaction throughput.

In the Sections \ref{sec:x-axis}, \ref{sec:y-axis} and \ref{sec:z-axis} we
describe the single axes with aid of examples, we argue where Ethereum is placed
with respect to the specific axis and we describe some solutions proposed to
augment the scalability of the system.
